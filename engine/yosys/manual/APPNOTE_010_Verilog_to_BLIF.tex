
% IEEEtran howto:
% http://ftp.univie.ac.at/packages/tex/macros/latex/contrib/IEEEtran/IEEEtran_HOWTO.pdf
\documentclass[9pt,technote,a4paper]{IEEEtran}

\usepackage[T1]{fontenc}   % required for luximono!
\usepackage[scaled=0.8]{luximono}  % typewriter font with bold face

% To install the luximono font files:
% getnonfreefonts-sys --all        or
% getnonfreefonts-sys luximono
%
% when there are trouble you might need to:
% - Create /etc/texmf/updmap.d/99local-luximono.cfg
%   containing the single line: Map ul9.map
% - Run update-updmap followed by mktexlsr and updmap-sys
%
% This commands must be executed as root with a root environment
% (i.e. run "sudo su" and then execute the commands in the root
% shell, don't just prefix the commands with "sudo").

\usepackage[unicode,bookmarks=false]{hyperref}
\usepackage[english]{babel}
\usepackage[utf8]{inputenc}
\usepackage{amssymb}
\usepackage{amsmath}
\usepackage{amsfonts}
\usepackage{units}
\usepackage{nicefrac}
\usepackage{eurosym}
\usepackage{graphicx}
\usepackage{verbatim}
\usepackage{algpseudocode}
\usepackage{scalefnt}
\usepackage{xspace}
\usepackage{color}
\usepackage{colortbl}
\usepackage{multirow}
\usepackage{hhline}
\usepackage{listings}
\usepackage{float}

\usepackage{tikz}
\usetikzlibrary{calc}
\usetikzlibrary{arrows}
\usetikzlibrary{scopes}
\usetikzlibrary{through}
\usetikzlibrary{shapes.geometric}

\lstset{basicstyle=\ttfamily,frame=trBL,xleftmargin=2em,xrightmargin=1em,numbers=left}

\begin{document}

\title{Yosys Application Note 010: \\ Converting Verilog to BLIF}
\author{Claire Xenia Wolf \\ November 2013}
\maketitle

\begin{abstract}
Verilog-2005 is a powerful Hardware Description Language (HDL) that can be used
to easily create complex designs from small HDL code. It is the preferred
method of design entry for many designers\footnote{The other half prefers VHDL,
a very different but -- of course -- equally powerful language.}.

The Berkeley Logic Interchange Format (BLIF) \cite{blif} is a simple file format for
exchanging sequential logic between programs. It is easy to generate and
easy to parse and is therefore the preferred method of design entry for
many authors of logic synthesis tools.

Yosys \cite{yosys} is a feature-rich
Open-Source Verilog synthesis tool that can be used to bridge the gap between
the two file formats. It implements most of Verilog-2005 and thus can be used
to import modern behavioral Verilog designs into BLIF-based design flows
without dependencies on proprietary synthesis tools.

The scope of Yosys goes of course far beyond Verilog logic synthesis. But
it is a useful and important feature and this Application Note will focus
on this aspect of Yosys.
\end{abstract}

\section{Installation}

Yosys written in C++ (using features from C++11) and is tested on modern Linux.
It should compile fine on most UNIX systems with a C++11 compiler. The README
file contains useful information on building Yosys and its prerequisites.

Yosys is a large and feature-rich program with a couple of dependencies. It is,
however, possible to deactivate some of the dependencies in the Makefile,
resulting in features in Yosys becoming unavailable. When problems with building
Yosys are encountered, a user who is only interested in the features of Yosys
that are discussed in this Application Note may deactivate {\tt TCL}, {\tt Qt}
and {\tt MiniSAT} support in the {\tt Makefile} and may opt against building
{\tt yosys-abc}.

\bigskip

This Application Note is based on GIT Rev. {\tt e216e0e} from 2013-11-23 of
Yosys \cite{yosys}. The Verilog sources used for the examples are taken from
yosys-bigsim \cite{bigsim}, a collection of real-world designs used for
regression testing Yosys.

\section{Getting Started}

We start our tour with the Navr\'e processor from yosys-bigsim. The Navr\'e
processor \cite{navre} is an Open Source AVR clone. It is a single module ({\tt
softusb\_navre}) in a single design file ({\tt softusb\_navre.v}). It also is
using only features that map nicely to the BLIF format, for example it only
uses synchronous resets.

Converting {\tt softusb\_navre.v} to {\tt softusb\_navre.blif} could not be
easier:

\begin{figure}[H]
\begin{lstlisting}[language=sh]
yosys -o softusb_navre.blif -S softusb_navre.v
\end{lstlisting}
 \renewcommand{\figurename}{Listing}
\caption{Calling Yosys without script file}
\end{figure}

Behind the scenes Yosys is controlled by synthesis scripts that execute
commands that operate on Yosys' internal state. For example, the {\tt -o
softusb\_navre.blif} option just adds the command {\tt write\_blif
softusb\_navre.blif} to the end of the script. Likewise a file on the
command line -- {\tt softusb\_navre.v} in this case -- adds the command
{\tt read\_verilog softusb\_navre.v} to the beginning of the
synthesis script. In both cases the file type is detected from the
file extension.

Finally the option {\tt -S} instantiates a built-in default synthesis script.
Instead of using {\tt -S} one could also specify the synthesis commands
for the script on the command line using the {\tt -p} option, either using
individual options for each command or by passing one big command string
with a semicolon-separated list of commands. But in most cases it is more
convenient to use an actual script file.

\section{Using a Synthesis Script}

With a script file we have better control over Yosys. The following script
file replicates what the command from the last section did:

\begin{figure}[H]
\begin{lstlisting}[language=sh]
read_verilog softusb_navre.v
hierarchy
proc; opt; memory; opt; techmap; opt
write_blif softusb_navre.blif
\end{lstlisting}
 \renewcommand{\figurename}{Listing}
\caption{\tt softusb\_navre.ys}
\end{figure}

The first and last line obviously read the Verilog file and write the BLIF
file.

\medskip

The 2nd line checks the design hierarchy and instantiates parametrized
versions of the modules in the design, if necessary. In the case of this
simple design this is a no-op. However, as a general rule a synthesis script
should always contain this command as first command after reading the input
files.

\medskip

The 3rd line does most of the actual work:

\begin{itemize}
\item The command {\tt opt} is the Yosys' built-in optimizer. It can perform
some simple optimizations such as const-folding and removing unconnected parts
of the design. It is common practice to call opt after each major step in the
synthesis procedure. In cases where too much optimization is not appreciated
(for example when analyzing a design), it is recommended to call {\tt clean}
instead of {\tt opt}.
\item The command {\tt proc} converts {\it processes} (Yosys' internal
representation of Verilog {\tt always}- and {\tt initial}-blocks) to circuits
of multiplexers and storage elements (various types of flip-flops).
\item The command {\tt memory} converts Yosys' internal representations of
arrays and array accesses to multi-port block memories, and then maps this
block memories to address decoders and flip-flops, unless the option {\tt -nomap}
is used, in which case the multi-port block memories stay in the design
and can then be mapped to architecture-specific memory primitives using
other commands.
\item The command {\tt techmap} turns a high-level circuit with coarse grain
cells such as wide adders and multipliers to a fine-grain circuit of simple
logic primitives and single-bit storage elements. The command does that by
substituting the complex cells by circuits of simpler cells. It is possible
to provide a custom set of rules for this process in the form of a Verilog
source file, as we will see in the next section.
\end{itemize}

Now Yosys can be run with the filename of the synthesis script as argument:

\begin{figure}[H]
\begin{lstlisting}[language=sh]
yosys softusb_navre.ys
\end{lstlisting}
 \renewcommand{\figurename}{Listing}
\caption{Calling Yosys with script file}
\end{figure}

\medskip

Now that we are using a synthesis script we can easily modify how Yosys
synthesizes the design. The first thing we should customize is the
call to the {\tt hierarchy} command:

Whenever it is known that there are no implicit blackboxes in the design, i.e.
modules that are referenced but are not defined, the {\tt hierarchy} command
should be called with the {\tt -check} option. This will then cause synthesis
to fail when implicit blackboxes are found in the design.

The 2nd thing we can improve regarding the {\tt hierarchy} command is that we
can tell it the name of the top level module of the design hierarchy. It will
then automatically remove all modules that are not referenced from this top
level module.

\medskip

For many designs it is also desired to optimize the encodings for the finite
state machines (FSMs) in the design. The {\tt fsm} command finds FSMs, extracts
them, performs some basic optimizations and then generate a circuit from
the extracted and optimized description. It would also be possible to tell
the {\tt fsm} command to leave the FSMs in their extracted form, so they can be
further processed using custom commands. But in this case we don't want that.

\medskip

So now we have the final synthesis script for generating a BLIF file
for the Navr\'e CPU:

\begin{figure}[H]
\begin{lstlisting}[language=sh]
read_verilog softusb_navre.v
hierarchy -check -top softusb_navre
proc; opt; memory; opt; fsm; opt; techmap; opt
write_blif softusb_navre.blif
\end{lstlisting}
 \renewcommand{\figurename}{Listing}
\caption{{\tt softusb\_navre.ys} (improved)}
\end{figure}

\section{Advanced Example: The Amber23 ARMv2a CPU}

Our 2nd example is the Amber23 \cite{amber}
ARMv2a CPU. Once again we base our example on the Verilog code that is included
in yosys-bigsim \cite{bigsim}.

\begin{figure}[b!]
\begin{lstlisting}[language=sh]
read_verilog a23_alu.v
read_verilog a23_barrel_shift_fpga.v
read_verilog a23_barrel_shift.v
read_verilog a23_cache.v
read_verilog a23_coprocessor.v
read_verilog a23_core.v
read_verilog a23_decode.v
read_verilog a23_execute.v
read_verilog a23_fetch.v
read_verilog a23_multiply.v
read_verilog a23_ram_register_bank.v
read_verilog a23_register_bank.v
read_verilog a23_wishbone.v
read_verilog generic_sram_byte_en.v
read_verilog generic_sram_line_en.v
hierarchy -check -top a23_core
add -global_input globrst 1
proc -global_arst globrst
techmap -map adff2dff.v
opt; memory; opt; fsm; opt; techmap
write_blif amber23.blif
\end{lstlisting}
 \renewcommand{\figurename}{Listing}
\caption{\tt amber23.ys}
\label{aber23.ys}
\end{figure}

The problem with this core is that it contains no dedicated reset logic.
Instead the coding techniques shown in Listing~\ref{glob_arst} are used to
define reset values for the global asynchronous reset in an FPGA
implementation. This design can not be expressed in BLIF as it is. Instead we
need to use a synthesis script that transforms this form to synchronous resets that
can be expressed in BLIF.

(Note that there is no problem if this coding techniques are used to model
ROM, where the register is initialized using this syntax but is never updated
otherwise.)

\medskip

Listing~\ref{aber23.ys} shows the synthesis script for the Amber23 core. In
line 17 the {\tt add} command is used to add a 1-bit wide global input signal
with the name {\tt globrst}. That means that an input with that name is added
to each module in the design hierarchy and then all module instantiations are
altered so that this new signal is connected throughout the whole design
hierarchy.

\begin{figure}[t!]
\begin{lstlisting}[language=Verilog]
reg [7:0] a = 13, b;
initial b = 37;
\end{lstlisting}
 \renewcommand{\figurename}{Listing}
\caption{Implicit coding of global asynchronous resets}
\label{glob_arst}
\end{figure}

\begin{figure}[b!]
\begin{lstlisting}[language=Verilog]
(* techmap_celltype = "$adff" *)
module adff2dff (CLK, ARST, D, Q);

parameter WIDTH = 1;
parameter CLK_POLARITY = 1;
parameter ARST_POLARITY = 1;
parameter ARST_VALUE = 0;

input CLK, ARST;
input [WIDTH-1:0] D;
output reg [WIDTH-1:0] Q;

wire [1023:0] _TECHMAP_DO_ = "proc";

wire _TECHMAP_FAIL_ =
    !CLK_POLARITY || !ARST_POLARITY;

always @(posedge CLK)
        if (ARST)
                Q <= ARST_VALUE;
        else
                Q <= D;

endmodule
\end{lstlisting}
\renewcommand{\figurename}{Listing}
\caption{\tt adff2dff.v}
\label{adff2dff.v}
\end{figure}

In line 18 the {\tt proc} command is called. But in this script the signal name
{\tt globrst} is passed to the command as a global reset signal for resetting
the registers to their assigned initial values.

Finally in line 19 the {\tt techmap} command is used to replace all instances
of flip-flops with asynchronous resets with flip-flops with synchronous resets.
The map file used for this is shown in Listing~\ref{adff2dff.v}. Note how the
{\tt techmap\_celltype} attribute is used in line 1 to tell the techmap command
which cells to replace in the design, how the {\tt \_TECHMAP\_FAIL\_} wire in
lines 15 and 16 (which evaluates to a constant value) determines if the
parameter set is compatible with this replacement circuit, and how the {\tt
\_TECHMAP\_DO\_} wire in line 13 provides a mini synthesis-script to be used to
process this cell.

\begin{figure*}
\begin{lstlisting}[language=C]
#include <stdint.h>
#include <stdbool.h>

#define BITMAP_SIZE 64
#define OUTPORT 0x10000000

static uint32_t bitmap[BITMAP_SIZE/32];

static void bitmap_set(uint32_t idx) { bitmap[idx/32] |= 1 << (idx % 32); }
static bool bitmap_get(uint32_t idx) { return (bitmap[idx/32] & (1 << (idx % 32))) != 0; }
static void output(uint32_t val) { *((volatile uint32_t*)OUTPORT) = val; }

int main() {
	uint32_t i, j, k;
	output(2);
	for (i = 0; i < BITMAP_SIZE; i++) {
		if (bitmap_get(i)) continue;
		output(3+2*i);
		for (j = 2*(3+2*i);; j += 3+2*i) {
			if (j%2 == 0) continue;
			k = (j-3)/2;
			if (k >= BITMAP_SIZE) break;
			bitmap_set(k);
		}
	}
	output(0);
	return 0;
}
\end{lstlisting}
\renewcommand{\figurename}{Listing}
\caption{Test program for the Amber23 CPU (Sieve of Eratosthenes). Compiled using
GCC 4.6.3 for ARM with {\tt -Os -marm -march=armv2a -mno-thumb-interwork
-ffreestanding}, linked with {\tt -{}-fix-v4bx} set and booted with a custom
setup routine written in ARM assembler.}
\label{sieve}
\end{figure*}

\section{Verification of the Amber23 CPU}

The BLIF file for the Amber23 core, generated using Listings~\ref{aber23.ys}
and \ref{adff2dff.v} and the version of the Amber23 RTL source that is bundled
with yosys-bigsim, was verified using the test-bench from yosys-bigsim.
It successfully executed the program shown in Listing~\ref{sieve} in the
test-bench.

For simulation the BLIF file was converted back to Verilog using ABC
\cite{ABC}. So this test includes the successful transformation of the BLIF
file into ABC's internal format as well.

The only thing left to write about the simulation itself is that it probably
was one of the most energy inefficient and time consuming ways of successfully
calculating the first 31 primes the author has ever conducted.

\section{Limitations}

At the time of this writing Yosys does not support multi-dimensional memories,
does not support writing to individual bits of array elements, does not
support initialization of arrays with {\tt \$readmemb} and {\tt \$readmemh},
and has only limited support for tristate logic, to name just a few
limitations.

That being said, Yosys can synthesize an overwhelming majority of real-world
Verilog RTL code. The remaining cases can usually be modified to be compatible
with Yosys quite easily.

The various designs in yosys-bigsim are a good place to look for examples
of what is within the capabilities of Yosys.

\section{Conclusion}

Yosys is a feature-rich Verilog-2005 synthesis tool. It has many uses, but
one is to provide an easy gateway from high-level Verilog code to low-level
logic circuits.

The command line option {\tt -S} can be used to quickly synthesize Verilog
code to BLIF files without a hassle.

With custom synthesis scripts it becomes possible to easily perform high-level
optimizations, such as re-encoding FSMs. In some extreme cases, such as the
Amber23 ARMv2 CPU, the more advanced Yosys features can be used to change a
design to fit a certain need without actually touching the RTL code.

\begin{thebibliography}{9}

\bibitem{yosys}
Claire Xenia Wolf. The Yosys Open SYnthesis Suite. \\
\url{https://yosyshq.net/yosys/}

\bibitem{bigsim}
yosys-bigsim, a collection of real-world Verilog designs for regression testing purposes. \\
\url{https://github.com/YosysHQ/yosys-bigsim}

\bibitem{navre}
Sebastien Bourdeauducq. Navr\'e AVR clone (8-bit RISC). \\
\url{http://opencores.org/project,navre}

\bibitem{amber}
Conor Santifort. Amber ARM-compatible core. \\
\url{http://opencores.org/project,amber}

\bibitem{ABC}
Berkeley Logic Synthesis and Verification Group. ABC: A System for Sequential Synthesis and Verification. \\
\url{http://www.eecs.berkeley.edu/~alanmi/abc/}

\bibitem{blif}
Berkeley Logic Interchange Format (BLIF) \\
\url{http://vlsi.colorado.edu/~vis/blif.ps}

\end{thebibliography}


\end{document}
